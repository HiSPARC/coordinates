\begin{abstract}

This is meant as documentation to describe the coordinate system and
units used in \hisparc data and analysis. We also have to deal with
other coordinate systems such as the one used in \corsika and some used
as intermediary in coordinate transformations.

\end{abstract}


\section{Introduction}

First several coordinate systems used by \hisparc will be discussed.
Including the units used and where it is used. Then some other
coordinate systems that we have to deal with are discussed, including
ways to convert between the different systems.


\section{Geographic}

\subsection{World Geodetic System 1984 (WGS84)}

To get the location of a station we use a \gps antenna. The \gps is
often not the center of a station, but is close. The \gps returns
coordinates in the WGS84 coordinate system. This defines a position with
a latitude, longitude and altitude. The latitude and longitude are
defined in decimal degrees, the altitude in meters. Latitude is positive towards West and longitude is positive towards North.


\subsection{Earth-Centered, Earth-Fixed (ECEF)}

...


\subsection{East, North, Up (ENU)}

East North Up is used in the code to easily get distances, relative
locations and angles between locations. It is defined relative to a
reference position. From the location east is positive x axis, north is
positive y axis and up (towards zenith) is the positive z axis. All
distances are in meters.


\subsection{Compass}

To measure the location of detectors in a station we need to define the
positions of detectors relative to the \gps antenna, since that position
is known. For this we use a simple system that requires a distance and
angle measurement. The distance between the \gps and a detector are
measured (in meters) and the angle relative to (True?) north is
determined by a compass. This angle is in degrees and goes from North to
East..


\section{Celestial}

\subsection{Azimuth and Zenith}

When a station detects a shower we try to reconstruct the direction of
its origin. The direction of a shower is then given by an azimuth and
zenith coordinate. These coordinates define a point on the semi-sphere
that is the sky above the detection station. The zenith is the point
directly above the observer. The zenith angle is the angle between the
direction and the zenith point. The azimuth is the direction in the
horizontal plane, it starts at East then goes to North (ENWS).

We do not expect nor consider air showers from below the horizon, so the
zenith angles, defined in radians, are and angle in the range (0,
$\frac{\pi}{2}$). The azimuth is restricted to the range ($-\pi$, $\pi$).


\subsection{Horizontal}

This is a system used as intermediary for some coordinate conversions.
It uses azimuth and altitude to define a direction. The altitude is the
opposite of the zenith, so 0 is horizontal and $\frac{\pi}{2}$ is the
zenith. The azimuth definition also differs, it moves from North to East
(NESW).


\subsection{Equatorial (J2000)}

Vernal equinox, epoch.
Right ascension, Declination, J2000
In decimal hours.


\section{\corsika}

\subsection{Geographic}

\corsika defines positions on the ground (or observation level) relative
to the core of the shower(??). Positive x axis points to magnetic North,
positive y axis to the West, and the z axis upwards.


\subsection{Celestial}

As origin of the shower \corsika looks from the point of view of the
shower. The $\theta$ angle is defined the same as our definition of
zenith, \SI{0}{\radian} is a shower from the zenith and
\SI{\pi / 2}{\radian} is a horizontal shower. The $\phi$ angle is
defined differently than our definition of azimuth. First it is the
angle the shower is heading towards. \SI{0}{\radian} is a shower heading
towards North, so coming form South, which we would define as
\SI{\pi / 2}{\radian}. The (positive) rotation of the angle is
in the same direction, from North to West. So the conversion is:.


\bibliography{References}

\begin{thebibliography}{widest-label}
\bibitem{corsika}
Corsika manual Chapter 6 Coordinate System.

\end{thebibliography}
