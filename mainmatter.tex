\section{Introduction}

Since we have to work with many coordinate systems it can be hard to
keep track of the definitions of each. This document is meant as a
reference to easily find the relations between the different systems.
First coordinate systems used by \hisparc will be discussed. Including
the units that are used and where it is used. Then other coordinate
systems that we have to deal with are discussed, including ways to
convert from those to our usual coordinate systems.

In \secref{sec:geographic} we discuss geographic coordinates to define
the coordinates of the observer. In \secref{sec:time} we discuss
different time keeping systems and how we can relate a time of
observation to the rotation of the Earth w.r.t the celestial sphere. The
systems to define a point on the celestial sphere are described in
\secref{sec:celestial}. In \secref{sec:corsika} the coordinate systems
which are used in \corsika simulations are described. Finally in
\secref{sec:example} an example shows the process to go from a \hisparc
detection to equatorial coordinates.

Make sure that the angles are converted to the correct units for the
cosine and sine functions when converting between coordinate systems.
The use of radians or degrees depends on the programming language.


\section{Geographic}
\label{sec:geographic}

Geographic coordinate systems define a point on the Earth. For \hisparc
two systems are used: WGS84 and ENU. \figref{fig:wgs84_ecef_enu}
illustrates the relationships between these systems. Additionally a
compass based system is used to determine detector locations relative to
the \gps antenna. These systems are described in the following sections.
The conversion formulae are taken from \cite[sec.~K]{usno:2014aa}.

\begin{figure}
    \centering
    \tikzsetnextfilename{externalized-wsg84_ecef_enu}

\tdplotsetmaincoords{70}{95}

\pgfmathsetmacro{\rlen}{.8}
\pgfmathsetmacro{\latitude}{90-45}  % phi
\pgfmathsetmacro{\longitude}{50}  % lambda

\begin{tikzpicture}[scale=5,thick,tdplot_main_coords]

    \coordinate (O) at (0,0,0);
    \tdplotsetcoord{P}{\rlen}{\latitude}{\longitude};

    % base
    \draw[base] (O) circle (\rlen);
    \draw[base] (\rlen,-.2,0) node[anchor=north east,rotate=-10] {Equator};
    \tdplotsetthetaplanecoords{90};
    \draw[base,tdplot_rotated_coords] (O) circle (\rlen);
    \tdplotsetthetaplanecoords{0};
    \draw[base,tdplot_rotated_coords] (\rlen,0,0) arc (0:90:\rlen)
        node[above right,rotate=88] {Prime Meridian};
    \tdplotsetthetaplanecoords{\longitude};
    \draw[base,dashed,tdplot_rotated_coords] (\rlen,0,0) arc (0:90:\rlen);
    \draw[base,dashed,tdplot_rotated_coords] (0,0,0) -- (0,\rlen,0);
%    \draw[base,dashed] (P) -- (Pxy);

    % ECEF
    \draw[ecef,-stealth] (O) -- (1,0,0) node[anchor=north]{$X$};
    \draw[ecef,-stealth] (O) -- (0,1,0) node[anchor=west]{$Y$};
    \draw[ecef,-stealth] (O) -- (0,0,1) node[anchor=south]{$Z$};
    \draw[-stealth] (O) -- (P);

    % WGS84
    \tdplotdrawarc[wgs84,-stealth]
        {(O)}{0.25}{0}{\longitude}{anchor=north}{$\lambda$};
    \tdplotsetthetaplanecoords{\longitude};
    \tdplotdrawarc[wgs84,-stealth,tdplot_rotated_coords]
        {(0,0,0)}{0.25}{90}{\latitude}{anchor=south west}{$\varphi$};

    % ENU
    \tdplotsetrotatedcoords{\longitude}{\latitude}{0};
    \tdplotsetrotatedcoordsorigin{(P)};

%   This should be the ENU axes,but I cant find the correct rotation:
%    \draw[enu,tdplot_rotated_coords,->] (0,0,0) -- (.3,0,0) node[anchor=west]{$East$};
%    \draw[enu,tdplot_rotated_coords,->] (0,0,0) -- (0,.3,0) node[anchor=south]{$North$};
%    \draw[enu,tdplot_rotated_coords,->] (0,0,0) -- (0,0,.4) node[anchor=south]{$Up$};
    \draw[base,tdplot_rotated_coords] (0,0,0) circle (.17);

    \draw[enu,-stealth,tdplot_rotated_coords] (0,0,0) -- (-.45,0,0)
        node[anchor=south]{North};
    \draw[enu,-stealth,tdplot_rotated_coords] (0,0,0) -- (0,.45,0)
        node[anchor=west]{East};
    \draw[enu,-stealth,tdplot_rotated_coords] (0,0,0) -- (0,0,.45)
        node[anchor=south]{Up};

\end{tikzpicture}

    \caption{Relationship between the WGS84 (orange), ECEF (blue) and ENU
             (green) coordinate systems. The Prime meridian is the IERS
             Reference Meridian.}
    \label{fig:wgs84_ecef_enu}
\end{figure}


\subsection{World Geodetic System 1984 (WGS84)}

The location of a \hisparc station is determined by means of a \gps
antenna (one antenna for each station). The \gps returns coordinates in
the WGS84 coordinate system. This defines a position with a latitude
($\varphi$), longitude ($\lambda$) and altitude ($h$). The latitude and
longitude are defined in decimal degrees, the altitude in meters.
Latitude is the angular distance between a location and the equator,
angles towards north are positive. The longitude is the angular distance
of a location to the Reference Meridian (from now on referred to as the
prime meridian) defined by the International Earth Rotation and
Reference Systems Service (IERS), angles towards east are positive. The
altitude is the height above an ellipsoid that approximates the shape of
the Earth. The parameters which define this ellipsoid are given in
\eqref{eq:wgs84} \cite{nga:2014aa}.


\subsection{Earth-Centered, Earth-Fixed (ECEF)}

The ECEF coordinate system is used as intermediate between WGS84 and
ENU. ECEF has its origin as the center of the Earth (Earth-Centered),
and its axes are aligned with the WGS84 system. The z-axis points to
North. The x-axis crosses the Earth surface where the equator
(\SI{0}{\degree} latitude) intersects the prime meridan (\SI{0}{\degree}
longitude). The y-axis intersects the equator at \SI{90}{\degree}
longitude. The axes rotate with the Earth (Earth-Fixed). A position is
defined by the $X$, $Y$, and $Z$ coordinates, each given in meters.

The formulae to convert WGS84 coordinates to ECEF coordinates are:

\begin{equation}
    \begin{array}{l}
        X = \left(N + h\right) \cos{\varphi} \cos{\lambda} \ , \\
        Y = \left(N + h\right) \cos{\varphi} \sin{\lambda} \ , \\
        Z = \left(\frac{b^2}{a^2} N + h\right) \sin{\varphi} \ .
    \end{array}
\end{equation}

\noindent
For the paramaters that define the ellipsoidal approximation of the
Earth surface we have

\begin{equation}
    \label{eq:wgs84}
    \begin{array}{l}
        a = \SI{6378137.0}{\meter} \ , \\
        f = \frac{1}{298.257223563} \ , \\
        b = a (1 - f) \ , \\
        e = \sqrt{2 f - f^2} \ , \\
        N = \frac{a}{\sqrt{1 - e^2 \sin^2 \varphi}} \ ,
    \end{array}
\end{equation}

\noindent here $a$ is the semi-major axis, $f$ the flattening, $b$ the
semi-minor axis and $e$ the eccentricity of the ellipsoid. $N$ is the
Normal, which is the distance between a location on the ellipsoid and
the intersection of its normal and the z-axis of the ellipsoid. For a
position on the equator (i.e. $\varphi = \SI{0}{\degree}$) $N = a$.

\begin{figure}
    \centering
    \tikzsetnextfilename{externalized-ellipsoid}

\pgfmathsetmacro{\alen}{1.4}  % equatorial radius ellipse
\pgfmathsetmacro{\blen}{1.2}  % polar radius ellipse (actual: 99.7% of a)
\pgfmathsetmacro{\llen}{.2}  % radius of latitude arc
\pgfmathsetmacro{\latitude}{50}  % phi
\pgfmathsetmacro{\reverselatitude}{-180+\latitude}  % phi

\begin{tikzpicture}[scale=2,thick]

    \definecolor{base}{RGB}{128,128,128}; % grey
    \definecolor{wgs84}{RGB}{255,153,0}; % orange
    \definecolor{ecef}{RGB}{13,98,153}; % blue
    \definecolor{enu}{RGB}{0,184,0}; % green
    \definecolor{compass}{RGB}{125,10,255}; % purple
    \definecolor{equa}{RGB}{253,39,39}; % red
    \definecolor{time}{RGB}{15,171,135}; % teal

    \coordinate (O) at (0,0);
    \coordinate (D) at (\latitude:\alen\space and \blen);

    % base
    \draw[base,name path=earth] (O) circle [x radius=\alen, y radius=\blen];
    \draw[base,name path=equator]
        (-\alen,0) -- node[near start,anchor=north] {$a$}
        (\alen,0) node[anchor=west] {Equator};
    \draw[base,name path=zaxis] (0,-\blen) --
        node[near start,anchor=east] {$b$} (0,\blen);

    % ECEF
    \draw[ecef,-stealth] (O) -- (0,\alen)
        node[anchor=south] {$Z$};

    % observer
%    \draw[color=black,->,-stealth] (O) -- (\latitude:\alen);
    \draw[enu,-stealth,name path=Up] (D) -- ++(\latitude:.3)
        node[above] {Up};
    \draw[enu,-stealth,name path=North] (D) -- ++(\latitude+90:.3)
        node[above] {North};
    \draw[color=black,stealth-,name path=Normal]
        (D) node[right] {Observer} -- node[right] {$N$}
        ++(\reverselatitude:\alen);

    \fill[base,name intersections={of=Normal and zaxis, by=Q}]
        (Q) circle (.02);

    \draw[wgs84,name intersections={of=Normal and equator, by=I}]
        (I)++(\llen,0) arc (0:\latitude:\llen) node[anchor=west] {$\varphi$};

\end{tikzpicture}

    \caption{Ellipsoidal approximation of Earth. Showing the definitions
             of the semi-major axis, the semi-minor axis, and the Normal.}
    \label{fig:ellipsoid}
\end{figure}


\subsection{East, North, Up (ENU)}

East, North, Up is used to obtain relative locations and angles with
respect to a reference position. It is a local coordinate system in a
plane tangential to the WGS84 elipsoid. From the reference position the
positive x-axis (East) is in the eastern direction, the positive y-axis
(North) is towards north, and the positive z-axis (Up) is towards the
Zenith. All distances are in meters.

The formulae to convert ECEF to ENU with a given reference position are:

\begin{equation}
    \begin{bmatrix}
        x \\
        y \\
        z
    \end{bmatrix}
    =
    \begin{bmatrix}
                        -\sin(\lambda_r) &                  \cos(\lambda_r) &              0. \\
        -\sin(\varphi_r) \cos(\lambda_r) & -\sin(\varphi_r) \sin(\lambda_r) & \cos(\varphi_r) \\
         \cos(\varphi_r) \cos(\lambda_r) &  \cos(\varphi_r) \sin(\lambda_r) & \sin(\varphi_r)
    \end{bmatrix}
    \begin{bmatrix}
        X - X_r \\
        Y - Y_r \\
        Z - Z_r
    \end{bmatrix}
\end{equation}

\noindent where $(X, Y, Z)$ and $(x, y, z)$ are the location in ECEF and
ENU respectively. $(\varphi_r, \lambda_r)$ and $(X_r, Y_r, Z_r)$ are the
reference position in WGS84 and ECEF respectively.


\subsection{`Compass'}

For reconstruction of events the precise location of the individual
detectors in each station is required. The location of detectors are
taken relative to the \gps antenna, because that position is known. We
use a system that requires a distance and angle measurement, see
\figref{fig:enu_compass}. The distance ($r$) between the \gps and a
detector (center of the scintillator) is measured (in meters), the angle
($\alpha$) relative to the geographic north pole (not magnetic north!)
is determined using a compass and the magnetic declination. This angle
is in degrees and goes from North to East (NESW). The height difference
($z$) between the \gps and detectors can also be measured, however, this
is only important if the detectors are not all at the same altitude or
are at a very different altitude from the \gps. The rotation angle of
the detector itself ($\beta$) is defined as the angle of the long side
of the detector relative to the North Pole (positive towards East). 

When a compass is used the angle needs to be compensated for the
difference between magnetic and geographic north, the difference is
called the magnetic declination. The magnetic declination is date and
location dependent \cite{canada:2013aa}. The magnetic declination can be
up to \SI{3}{\degree} for some \hisparc station locations.

\begin{figure}
    \centering
    \tikzsetnextfilename{externalized-enu_compass}

\pgfmathsetmacro{\scale}{.2}
\pgfmathsetmacro{\dw}{.5*\scale}
\pgfmathsetmacro{\dh}{1*\scale}

\begin{tikzpicture}[scale=2,thick]

    \definecolor{enu}{RGB}{0,184,0}; % green
    \definecolor{compass}{RGB}{125,10,255}; % purple
    \definecolor{base}{RGB}{128,128,128}; % grey

    % coordinates
    \coordinate (O) at (0,0);

    % Draw ENU axes
    \draw[enu,-stealth] (O) -- (0,2) node[anchor=south] {North};
    \draw[enu,-stealth] (O) -- (2,0) node[anchor=west] {East};

    % Draw GPS
    \fill[base] (O) circle[radius=.05] node[anchor=north east] {\gps};

    % Draw a detectors
    \foreach \num / \col / \alphavec / \rlen / \betavec in {0/black/315/8.97/135,
                                                            1/red/315/3.15/135,
                                                            2/green/225/5.09/45,
                                                            3/blue/45/4.89/45} {
        \coordinate (d\num) at (90-\alphavec:\rlen*\scale);
        \fill[color=\col,rotate=90-\betavec] (d\num) ++(-\dh/2,-\dw/2) rectangle +(\dh,\dw);
    }
%    \fill[base,rotate=\betavec] (d0) ++(-\dh/2,-\dw/2) rectangle +(\dh,\dw);

    % Draw distance line
    \draw[compass,-stealth] (O) -- node[below right] {$r$} (d3);

    % Draw alpha line
    \draw[compass,-stealth] (0,.5) arc (90:90-45:.5);
    \draw[compass] (70:.6) node {$\alpha$};

    % Draw North from detector
    \draw[-stealth] (d0) -- +(0,.5) node[anchor=south] {N};
    % Draw detector angle
    \draw[-stealth] (d0) -- +(90-135:.5);

    % Draw beta line
    \draw[compass,-stealth] (d0) ++(0,.3) arc (90:90-135:.3);
    \draw[compass] (d0) ++(90-60:.33) node[anchor=south] {$\beta$};

\end{tikzpicture}

    \caption{Relationship between the ENU (green) and Compass (purple)
             coordinate system. The detectors of station 503 are shown.
             Its \gps is used as the origin for the ENU coordinates. The
             $\alpha$ and $r$ coordinate of detector 4 (blue) and the
             $\beta$ coordinate of detector 1 (black) are illustrated.
             The values are given in \secref{sec:example}.}
    \label{fig:enu_compass}
\end{figure}

The formulae to convert the location of a station in Compass coordinates
to ENU are:

\begin{equation}
    \begin{array}{l}
        x = r * \sin{\alpha} \ , \\
        y = r * \cos{\alpha} \ , \\
        z = z \ .
    \end{array}
\end{equation}


\section{Time}
\label{sec:time}

For \hisparc events it is important to know precisely when they occur, a
\gps is used to get synchronized and precise timing. It gives us a
unique timestamp, this needs to be converted to Local Sidereal Time for
the conversion of celestial coordinates in \secref{sec:celestial}. The
full transformation chain is as follows: GPS $\to$ UTC $\to$ JD $\to$
GMST $\to$ LST.


\subsection{\gps time}

\gps time started counting on 6 January 1980 00:00 UTC (Coordinated
Universal Time), i.e. \num{315964800} UNIX seconds after the UNIX epoch
of 1 January 1970 00:00 UTC. \gps time is a continuously increasing
number. This is different from UTC time in which certain seconds are
repeated when `leap seconds' are added \cite{usno:2012aa}. Currently (6
February 2015) the difference between UTC and \gps time is 16 seconds.
From the \hisparc electronics we get a \gps timestamp, adjusted to the
UNIX epoch, i.e. $\mathrm{GPS} + \mathrm{UNIX(GPS=0)}$. The timestamp is
given in nanoseconds. The \gps timestamp is therefore a large number
which requires 64-bits to be represented digitally. For instance the
\gps timestamp for 1 December 2014 at 00:00:29.886222166 (\gps time) is

\begin{equation}
    \SI{1101427229886222166}{\nano\second} +
        \SI{315964800000000000}{\nano\second} =
        \SI{1417392029886222166}{\nano\second} \ .
\end{equation}


\subsection{UTC time}

Similar to \gps but corrected with leap seconds to keep it more in sync
with solar time. A leap second means that a second is counted twice. The
IERS decides when to add leap seconds. \hisparc does not use UTC time
because of the duplicate timestamps, which makes it possible for a
station to detect two events at the `same time'. Moreover, it causes
issues when looking for coincidences between stations.


\subsubsection{Leap seconds}

Leap seconds can be introduced at the end of either 30 June or 31
December, the leap second is represented in UTC time as 23:59:60.
\tabref{table:leapseconds} gives a list of the recent leap seconds,
which are important for \hisparc. Leap seconds are announced about 6
months before they take effect.

\begin{table}
    \centering
    \begin{tabular}{ l c }
        \toprule
        Date & Leap seconds \\
        \midrule
        31 December, 1998 & 13 \\
        31 December, 2005 & 14 \\
        31 December, 2008 & 15 \\
        30 June, 2012 & 16 \\
        30 June, 2015 & 17 \\
        \bottomrule
    \end{tabular}
   \caption{Leap seconds in effect or introduced after 2002. This is the
            amount of seconds that \gps time is ahead of UTC time.}
   \label{table:leapseconds}
\end{table}


\subsection{Julian Date (JD)}

JD is a decimal number counting the number of days since 1 January 4713
B.C. 12:00:00. Calculating this is not trivial since one has to account
for the fact that the 5th up to and including the 14th of October in
1582 do not exist. Those ten days were skipped because the Julian
calendar, which was used before that gap, introduced too many leap
years, every year divisible by 4 was a leap year. The Gregorian calendar
used afterwards compensated by having less leap years. Leap years still
occur when years are divisible by 4, except when they are also divisible
by 100 (e.g. 1900), except when they are also divisible by 400 (e.g.
2000) \cite{acf:2014aa}. Consider these examples: 1999 is not divisible
by 4 so not a leap year, 2004 is divisible by 4 and not by 100 and is a
leap year, 1900 is divisible by 4 and 100 but not 400 so is not a leap
year, and 2000 is divisible by 4, 100 and 400 and is therefore a leap
year.

The following conversion is valid for dates after 15 October 1582. The
brackets $\floor{\ }$ indicate that the decimal part of the value should
be discarded. The $\mathit{month}$ starts at 1 for January and
$\mathit{hour}$ is the decimal number of hours.

\begin{equation}
    \begin{array}{l}
        a = \floor{\frac{14 - \mathit{month}}{12}} \ , \\
        A = \floor{\frac{\mathit{year - a}}{100}} \ , \\
        B = 2 - A + \floor{\frac{A}{4}} \ , \\
        C = \floor{365.25 (\mathit{year} - a)} \ , \\
        D = \floor{30.6001 (\mathit{month} + 12 a + 1)} \ , \\
        \mathit{JD} = B + C + D + \mathit{day} +
                      \frac{\mathit{hour}}{24} + 1720994.5 \ .
    \end{array}
\end{equation}


\subsubsection{Modified Julian Date (MJD)}

Modified Julian Date to keep number shorter uses epoch 17 November 1858
00:00:00. The difference between a Modified Julian Date and a Julian
Date is 2400000.5 days. The .5 days moves the reference time from noon
to midnight.

\begin{equation}
    \mathit{MJD} = \mathit{JD} - 2400000.5 \ .
\end{equation}


\subsection{Sidereal Time}

In sidereal time the rotation of Earth relative to the `fixed'
background stars is taken. A sidereal day is a full rotation of Earth
around its axis, relative to background stars. In a solar day the Earth
rotates more than \SI{360}{\degree} around its axis because it orbits
around the Sun, causing the position of the Sun relative to the Earth to
change. The Earth orbits the Sun every 365.25 days, so in one day the
orbit proceeds by $\frac{\SI{360}{\degree}}{\SI{365.25}{\day}} \approx
\SI{1}{\degree\per\day}$. So an extra rotation of approximately
\SI{1}{\degree} around its axis is required to make the same part of the
Earth face the Sun (a solar day). A sidereal day is therefore around
\SI{4}{\minute} shorter than a solar day. In \figref{fig:sidereal_time}
the difference between a solar and sidereal day is illustrated.

\begin{figure}
    \centering
    \pgfmathsetmacro{\rsun}{.2}
\pgfmathsetmacro{\rearth}{.15}
\pgfmathsetmacro{\AU}{1.1}
\pgfmathsetmacro{\margin}{.05}

\pgfmathsetmacro{\orbit}{270}
\pgfmathsetmacro{\siderealday}{32}
\pgfmathsetmacro{\solarday}{50}
\pgfmathsetmacro{\omargin}{10}

\begin{tikzpicture}[scale=2,thick]

    \definecolor{base}{RGB}{128,128,128}; % grey
    \definecolor{sidereal}{RGB}{44,92,198}; % blue
    \definecolor{solar}{RGB}{202,112,27}; % orange
    \definecolor{earth}{RGB}{0,184,0}; % green

    % coordinates
    \coordinate (O) at (0,0);
    \coordinate (earth_0) at (\orbit:\AU);
    \coordinate (earth_1) at (\orbit+\siderealday:\AU);
    \coordinate (earth_2) at (\orbit+\solarday:\AU);

    % Earth
    \draw[base,dashed] (O) circle (\AU);
    \draw[earth] (250:\AU) node[anchor=north east] {Earth};
    \draw[earth,fill=green] (earth_0) circle (\rearth);
    \draw[earth,fill=green] (earth_1) circle (\rearth);
    \draw[earth,fill=green] (earth_2) circle (\rearth);

    % AU
    \draw[base,dashed] (O) -- (earth_0);
    \draw[base,dashed] (O) -- (earth_1);
    \draw[base,dashed] (O) -- (earth_2);

    % Sun
    \draw[solar,fill=yellow] (O) circle (\rsun);
    \draw[solar] (180:\rsun) node[anchor=north east] {Sun};

    % Prime Meridian
    \draw (earth_0) -- ($ (earth_0)!\rearth!(O) $);
    \draw (earth_1) -- ++(90:\rearth);
    \draw (earth_2) -- ($ (earth_2)!\rearth!(O) $);

    % Earth orbit
    \tdplotdrawarc[base,-stealth]
        {(O)}{\AU-\margin}{\orbit+\omargin}{\orbit+\siderealday-\omargin}
        {}{};
    \tdplotdrawarc[sidereal,-stealth]
        {(O)}{\AU+\rearth+\margin+\margin}{\orbit}{\orbit+\siderealday}
        {anchor=north}{Sidereal day};
    \tdplotdrawarc[solar,-stealth]
        {(O)}{\AU+\rearth+\margin}{\orbit}{\orbit+\solarday}
        {anchor=north west}{Solar day};

    % Earth rotation
    \tdplotdrawarc[base,-stealth]
        {(earth_0)}{\rearth-\margin}{110}{430}{anchor=west}{};

    % Earth rotation
    \tdplotdrawarc[base] {(O)}{\rsun+\margin}{\orbit}{\orbit+\siderealday}{}{};
    \tdplotdrawarc[base] {(earth_1)}{\rearth+\margin}{90}{90+\siderealday}{}{};


    % Stellar background
    \draw[sidereal,->] ($ (O) + (90:\rsun+\margin) $) -- ++(90:.2)
        node[anchor=south] {Background stars};
    \draw[sidereal,->] ($ (earth_0) + (90:\rearth+\margin) $) -- ++(90:.2);
    \draw[sidereal,->] ($ (earth_1) + (90:\rearth+\margin) $) -- ++(90:.2);
    \draw[sidereal,->] ($ (earth_2) + (90:\rearth+\margin) $) -- ++(90:.2);

\end{tikzpicture}

    \caption{This shows the orbit of Earth (green) around the Sun
             (yellow). The black line on the Earth represents the prime
             meridian. A full rotation of the Earth causes the prime
             meridian to point to the same background stars, this is a
             sidereal day. It takes longer, due to the orbit of the
             Earth, for the prime meridian to point to the Sun again,
             which is a solar day. Used sizes and angles are
             illustrative, not realistic.}
    \label{fig:sidereal_time}
\end{figure}


\subsubsection{Greenwich Mean Sidereal Time (GMST)}

GMST is the hour angle of the vernal equinox (\secref{sec:celestial})
with respect to the prime meridian at Greenwich, expressed in decimal
hours \cite{kaplan:2011aa}.

\begin{equation}
    \mathit{GMST} = 6.697374558 + 0.06570982441908\ \mathit{D0} +
                    1.00273790935\ H + 0.000026\ T^2 \ .
\end{equation}

Here $D0$ is the Julian Date of the previous midnight using the J2000
epoch (1 January, 2000 12:00 UTC or Julian date 2451545.0), $H$ is the
number of hours since the last midnight, and $T$ is the number of
centuries since J2000.

\begin{figure}
    \centering
    \pgfmathsetmacro{\rlen}{1.5}
\pgfmathsetmacro{\longitude}{80}  % lambda
\pgfmathsetmacro{\gmst}{45}  % Greenwich mean sidereal time
\pgfmathsetmacro{\lst}{\gmst+\longitude}  % local sidereal time in degrees

\begin{tikzpicture}[scale=2,thick]

    \definecolor{base}{RGB}{128,128,128}; % grey
    \definecolor{wgs84}{RGB}{255,153,0}; % orange
    \definecolor{side}{RGB}{13,98,153}; % blue
    \definecolor{enu}{RGB}{0,184,0}; % green
    \definecolor{compass}{RGB}{125,10,255}; % purple
    \definecolor{equa}{RGB}{253,39,39}; % red
    \definecolor{time}{RGB}{15,171,135}; % teal

    \coordinate (O) at (0,0);

    % base
    \draw[base] (O) circle (\rlen); % Equator
    \tdplotdrawarc[base,-stealth]
        {(O)}{.2}{20}{340}{anchor=east}{Rotation};

    % Prime Meridian
    \draw[base] (O) -- (0:\rlen)
        node[right] {Prime Meridian};

    % WGS84
    % longitude
    \tdplotdrawarc[wgs84,-stealth]
        {(O)}{0.5}{0}{\longitude}{anchor=north}{$\lambda$};

    % gmst
    \tdplotdrawarc[time,-stealth]
        {(O)}{\rlen-.3}{-\gmst}{0}{anchor=south east}{GMST};
    \tdplotdrawarc[side,-stealth]
        {(O)}{\rlen-.2}{-\gmst}{\lst-\gmst}{anchor=east}{LST};

    % observer
    \draw[color=black,-stealth] (O) -- (\longitude:\rlen)
        node[above right] {Observer};

    % Equatorial
    \draw[base,-stealth] (O) -- (-\gmst:\rlen+.5)
        node[anchor=north]{$\vernal$};

\end{tikzpicture}

    \caption{Top-down view of Earth showing the relation between GMST
             (teal) and LST (blue) for an observer at a longitude
             $\lambda$ (orange). The observer and prime meridian move with
             the rotation of the Earth, the vernal equinox does not.}
    \label{fig:wgs84_gmst_lst}
\end{figure}


\subsubsection{Local Sidereal Time (LST)}

Similar to GMST but takes observers longitude into account. To get the
Local Sidereal Time the observers longitude has to be added to the GMST.
To convert the longitude from degrees to hours it has to be divided by
15, $\frac{\SI{360}{\degree}}{\SI{24}{\hour}} =
\SI{15}{\degree\per\hour}$.

\begin{equation}
    \begin{array}{l}
        \mathit{LST} = \mathit{GMST} + \frac{\lambda}{15} \ .
    \end{array}
\end{equation}

The relation between LST and GMST is illustrated in
\figref{fig:wgs84_gmst_lst}.


\section{Celestial}
\label{sec:celestial}

Here we describe the different coordinate systems that are used to
define the direction (origin) of an air shower. The zenith and azimuth
coordinates rotate with the rotation of the Earth, because they are
anchored to the location of the observer. The Equatorial system is
linked to the celestial sphere and is independent from the rotation of
Earth, the time of observation, and the location of the observer. To
transform from zenith and azimuth to Equatorial the position of the
observer and time of observation are required. The Horizontal coordinate
system is used as intermediate, it also defines an azimuth coordinate,
this is different from the azimuth coordinate in the zenith and azimuth
system.


\subsection{Zenith and azimuth coordinates}

When a station detects a shower we try to reconstruct the direction of
its origin, i.e. the position on the sky the shower came from. This
direction is then given by a zenith ($\theta$) and azimuth ($\phi$)
coordinate. These coordinates define a point on the semi-sphere that is
the sky above the detection station. The Zenith is the point directly
above the observer. The zenith angle is the angle between the direction
and the Zenith point, so straight up is \SI{0}{\radian} and the horizon
\SI{\pi / 2}{\radian}. The azimuth is the direction in the horizontal
plane, it starts at East (\SI{0}{\radian}) then goes to North (ENWS).
This is illustrated in pink in \figref{fig:enu_horizontal}.

We neither expect nor consider air showers from below the horizon, so the
zenith angles, defined in radians, are an angle in the range [0,
$\frac{\pi}{2}$). The azimuth is restricted to the range [$-\pi$, $\pi$).

\begin{figure}
    \centering
    \tdplotsetmaincoords{70}{-10}

\pgfmathsetmacro{\rlen}{.8}
\pgfmathsetmacro{\zenith}{90-45}  % theta
\pgfmathsetmacro{\azimuth}{300}  % phi

\begin{tikzpicture}[scale=5,thick,tdplot_main_coords]

    \definecolor{base}{RGB}{128,128,128}; % grey
    \definecolor{wgs84}{RGB}{255,153,0}; % orange
    \definecolor{ecef}{RGB}{13,98,153}; % blue
    \definecolor{enu}{RGB}{0,184,0}; % green
    \definecolor{zenazi}{RGB}{200,20,200}; % pink
    \definecolor{hori}{RGB}{200,180,20}; % gold

    % coordinates
    \coordinate (O) at (0,0,0);
    \tdplotsetcoord{P}{\rlen}{\zenith}{\azimuth};

    % base
    \draw[base] (O) circle (\rlen);
    \draw[base] (-\rlen*.8,-\rlen*.8,0) node[anchor=west,rotate=-10] {Horizon};
    \tdplotsetthetaplanecoords{\azimuth};
    \draw[base,dashed,tdplot_rotated_coords] (\rlen,0,0) arc (0:90:\rlen);
    \draw[base,dashed,tdplot_rotated_coords] (0,0,0) -- (0,\rlen,0);
%    \draw[base,dashed] (P) -- (Pxy);

    % ENU
    \draw[enu,-stealth] (O) -- (1,0,0) node[anchor=west]{East};
    \draw[enu,-stealth] (O) -- (0,1,0) node[anchor=south east]{North};

    % Zenith Azimuth
    \tdplotdrawarc[zenazi,-stealth]
        {(O)}{0.2}{0}{\azimuth}{anchor=south east}{$\phi$};

    % Horizontal
    \tdplotdrawarc[hori,-stealth]
        {(O)}{0.4}{90}{\azimuth-360}{anchor=south west}{$A$};

    \tdplotsetthetaplanecoords{\azimuth};
    % Zenith Azimuth
    \tdplotdrawarc[zenazi,-stealth,tdplot_rotated_coords]
        {(0,0,0)}{0.25}{0}{\zenith}{anchor=south west}{$\theta$};
    % Horizontal
    \tdplotdrawarc[hori,-stealth,tdplot_rotated_coords]
        {(0,0,0)}{0.25}{90}{\zenith}{anchor=south west}{$a$};

    % ENU
    \draw[enu,-stealth] (O) -- (0,0,1) node[anchor=south]{Up};

    % star
    \draw[-stealth] (O) -- (P) node[anchor=south west] {$\star$};

\end{tikzpicture}

    \caption{Relationship between the ENU (green), Zenith Azimuth (pink)
             and horizontal (gold) coordinate systems.}
    \label{fig:enu_horizontal}
\end{figure}


\subsection{Horizontal coordinate system}

This is a system used as intermediary for some coordinate conversions.
It azimuth ($A$) and altitude ($a$) to define a direction. The
altitude is the complement of the zenith, so \SI{0}{\radian} is horizontal
and \SI{\pi / 2}{\radian} is the zenith. The azimuth definition also
differs, in Horizontal coordinates it moves from North to East (NESW).
This is illustrated in gold in \figref{fig:enu_horizontal}.

The formulae to convert from zenith ($\theta$) and azimuth ($\phi$) to
altitude ($a$) and azimuth ($A$) are:

\begin{equation}
    \begin{array}{l}
        a = \frac{\pi}{2} - \theta \ , \\
        A = \frac{\pi}{2} - \phi \ .
    \end{array}
\end{equation}

Conversions can cause values to go beyond the allowed range of values
for angles or times. They may have to be brought back into the range
after the conversion.


\subsection{Equatorial (J2000)}

\figref{fig:equatorial} shows the relation between the celestial sphere
and Equatorial coordinates. The right ascension ($\alpha$) is the angle
between the projection of the position on the celestial sphere ($\star$)
on the plane of the Celestial equator and the vernal equinox
($\vernal$). The declination ($\delta$) is the angle between the plane
of the Celestial equator and the sky position. Due to precession (change
of direction of Earth's rotational axis) the celestial positions slowly
change over time. To compensate for precession coordinates are
calculated as they would have been on a specific date, the J2000 epoch
is commonly used. The equatorial coordinates are fixed to the celestial
sphere and not influenced by the rotation of the Earth.

\begin{figure}
    \centering
    \tdplotsetmaincoords{70}{95}

\pgfmathsetmacro{\rlen}{.4}

\pgfmathsetmacro{\clen}{1.6}
\pgfmathsetmacro{\ra}{50}  % right ascension in degrees
\pgfmathsetmacro{\dec}{90-50}  % declination in degrees

\begin{tikzpicture}[scale=3,thick,tdplot_main_coords]

    \definecolor{base}{RGB}{128,128,128}; % grey
    \definecolor{wgs84}{RGB}{255,153,0}; % orange
    \definecolor{ecef}{RGB}{13,98,153}; % blue
    \definecolor{enu}{RGB}{0,184,0}; % green
    \definecolor{compass}{RGB}{125,10,255}; % purple
    \definecolor{equa}{RGB}{253,39,39}; % red
    \definecolor{time}{RGB}{15,171,135}; % teal

    \coordinate (O) at (0,0,0);
    \coordinate (top) at (0,0,\rlen);

    % base
    \draw[base] (O) circle (\rlen);
    \draw[base] (\rlen,-.2,0);  % Equator
    \tdplotdrawarc[time,-stealth]
        {(0,0,\rlen)}{.2}{20}{340}{anchor=west}{};

    \tdplotsetthetaplanecoords{90};
    \draw[base,tdplot_rotated_coords] (O) circle (\rlen);

    % celestial base
    \draw[base] (O) circle (\clen);
    \draw[base] (\clen,-.75,0)
        node[anchor=north,rotate=-10] {Celestial Equator};
    \draw[base,-stealth] (O) -- (0,0,\clen + .3)
        node[anchor=south] {Celestial North Pole};
    \tdplotsetthetaplanecoords{90};
    \draw[base,tdplot_rotated_coords] (O) circle (\clen);

    % Equatorial
    \draw[base,-stealth] (O) -- (\clen + .3,0,0)
        node[anchor=north]{$\vernal$};

    \tdplotsetcoord{S}{\clen}{\dec}{\ra};
    \draw[-stealth] (O) -- (S) node[anchor=south west] {$\star$};

    \tdplotsetthetaplanecoords{\ra};
    \draw[base,dashed,tdplot_rotated_coords] (\clen,0,0) arc (0:90:\clen);
    \draw[base,dashed,tdplot_rotated_coords] (0,0,0) -- (0,\clen,0);
    \tdplotdrawarc[equa,-stealth,tdplot_rotated_coords]
        {(0,0,0)}{\clen}{90}{\dec}{anchor=west}{$\delta$};
    \tdplotdrawarc[equa,-stealth]
        {(O)}{\clen}{0}{\ra}{anchor=north}{$\alpha$};

\end{tikzpicture}

    \caption{This figures shows the relation between a position on the
             sky (black), Equatorial coordinates (red), and the
             Celestial sphere.}
    \label{fig:equatorial}
\end{figure}

Given an observers position in WGS84, a time of observation in LST
(derived from the \gps timestamp and observer longitude) and horizontal
coordinates (converted from the zenith and azimuth coordinates) pointing
to the source the following formulae can be used to get the
corresponding equatorial coordinates \cite[p. 37]{duffet-smith:1990aa}.

\begin{equation}
    \label{eq:equatorial}
    \begin{array}{l}
        \delta = \arcsin{\left((\sin{a} \sin{\varphi}) +
                               (\cos{a} \cos{\varphi} \cos{A})\right)} \ , \\
        \mathit{HA} = \arccos{\left(\frac{\sin{a} - (\sin{\varphi} \sin{\delta})}
                                         {\cos{\varphi} \cos{\delta}}\right)} \ , \\
        \alpha = \mathit{LST} - \mathit{HA} \ .
    \end{array}
\end{equation}

The output range of $\arccos$ is [0, $\pi$), while that of $\mathit{HA}$ is
[0, $2\pi$). If the azimuth is positive use: $\mathit{HA} = 2 \pi -
\mathit{HA}$.

Where $\varphi$ the geographical latitude, $a$ the altitude, $A$ the
azimuth, $\mathit{LST}$ the Local sidereal time, $\delta$ is the
declination, $\alpha$ the right ascension and the intermediate variable
$\mathit{HA}$ is the hour angle. This is illustrated in
\figref{fig:wgs84_zenazi_lst_equatorial}.


\begin{figure}
    \centering
    \tikzsetnextfilename{externalized-wsg84_zenazi_lst_equatorial}

\tdplotsetmaincoords{70}{95}

\pgfmathsetmacro{\rlen}{1.2}
\pgfmathsetmacro{\latitude}{90-35}  % phi
\pgfmathsetmacro{\longitude}{100}  % lambda

\pgfmathsetmacro{\clen}{3}
\pgfmathsetmacro{\ra}{28}  % right ascension in degrees
\pgfmathsetmacro{\dec}{90-50}  % declination in degrees

\pgfmathsetmacro{\gmst}{-45}  % Greenwich mean sidereal time
\pgfmathsetmacro{\lst}{\longitude+\gmst}  % local sidereal time in degrees

%\pgfmathsetmacro{\zenith}{22}  % theta
%\pgfmathsetmacro{\azimuth}{0}  % phi

\begin{tikzpicture}[scale=2,thick,tdplot_main_coords]

    \coordinate (O) at (0,0,0);

    % base
    \draw[base] (O) circle (\rlen);
    \draw[base] (\rlen,-.2,0);  % Equator
    \tdplotsetthetaplanecoords{90};
    \draw[base,tdplot_rotated_coords] (O) circle (\rlen);
    \tdplotsetthetaplanecoords{\gmst};
    % Prime Meridian
    \draw[base,dashed,tdplot_rotated_coords] (0,0,0) -- (0,\rlen,0);
    \draw[base,tdplot_rotated_coords] (\rlen,0,0) arc (0:90:\rlen);
    \tdplotsetthetaplanecoords{\lst};
    \draw[base,dashed,tdplot_rotated_coords] (\rlen,0,0) arc (0:90:\rlen);
    \draw[base,dashed,tdplot_rotated_coords] (\clen,0,0) arc (0:90:\clen);
    \draw[base,dashed,tdplot_rotated_coords] (0,0,0) -- (0,\clen,0);

    \tdplotdrawarc[time,-stealth]
        {(0,0,\rlen)}{.2}{20}{340}{anchor=west}{};

    % celestial base
    \draw[base] (O) circle (\clen);
    \draw[base] (\clen,-.8,0)
        node[anchor=north east,rotate=-10] {Celestial Equator};
    \draw[base,-stealth] (O) -- (0,0,\clen + .3)
        node[anchor=south] {Celestial North Pole};
    \tdplotsetthetaplanecoords{90};
    \draw[base,tdplot_rotated_coords] (O) circle (\clen);

    % ECEF
%    \tdplotsetrotatedcoords{0}{0}{\gmst};
%    \draw[ecef,-stealth,tdplot_rotated_coords] (O) -- (\rlen+.2,0,0)
%        node[anchor=north] {$X$};
%    \draw[ecef,-stealth,tdplot_rotated_coords] (O) -- (0,\rlen+.2,0)
%        node[anchor=west] {$Y$};
%    \draw[ecef,-stealth,tdplot_rotated_coords] (O) -- (0,0,\rlen+.2)
%        node[anchor=west] {$Z$};

    % WGS84
    % longitude
    \tdplotsetrotatedcoords{0}{0}{\gmst};
    \tdplotdrawarc[wgs84,-stealth,tdplot_rotated_coords]
        {(O)}{0.4}{0}{\longitude}{anchor=north}{$\lambda$};
    % latitude
    \tdplotsetthetaplanecoords{\lst};
    \tdplotdrawarc[wgs84,-stealth,tdplot_rotated_coords]
        {(0,0,0)}{0.4}{90}{\latitude}{anchor=west}{$\varphi$};

    % gmst
    \tdplotsetthetaplanecoords{\lst};
    \tdplotdrawarc[time,-stealth]
        {(O)}{\clen}{\lst}{\ra}{anchor=north west}{HA}; %  Flip arrow to other side?
    \tdplotdrawarc[time,-stealth]
        {(O)}{\rlen}{0}{\lst}{anchor=north}{LST};

    % observer
    \tdplotsetcoord{P}{\rlen}{\latitude}{\lst};
    \tdplotsetcoord{P'}{\clen}{\latitude}{\lst};
    \draw[color=black,-stealth] (O) -- (P);
    \draw[base] (P) -- (P') node[anchor=south west] {Zenith};
%    \draw[base,dashed] (P) -- (Pxy);

    % Equatorial
    \draw[base,-stealth] (O) -- (\clen + .3,0,0)
        node[anchor=north]{$\vernal$};

    \tdplotsetcoord{S}{\clen}{\dec}{\ra};
    \draw[-stealth] (O) -- (S) node[anchor=south west] {$\star$};
%    \draw[base,dashed] (S) -- (Sxy);


    % Observer to Star
    \tdplotsetrotatedcoords{\lst}{\latitude}{0};
    \tdplotsetrotatedcoordsorigin{(P)};

    \tdplotdrawarc[zenazi,-stealth,tdplot_rotated_coords]
        {(0,0,0)}{0.2}{90}{225}{anchor=south east}{$\phi$};
%    \tdplotsetthetaplanecoords{\ra};
%    \tdplotdrawarc[zenazi,-stealth,tdplot_rotated_coords]
%        {(0,0,0)}{0.2}{90}{240}{anchor=west}{$\phi$};

    \draw[base,tdplot_rotated_coords] (0,0,0) circle (.25);
    \draw[color=black,-stealth] (P) -- (S);
%    \draw[enu,-stealth,tdplot_rotated_coords] (0,0,0) -- (-.5,0,0)
%        node[anchor=south]{N};
    \draw[enu,-stealth,tdplot_rotated_coords] (0,0,0) -- (0,.75,0)
        node[anchor=west]{E};
%    \draw[enu,-stealth,tdplot_rotated_coords] (0,0,0) -- (0,0,.35)
%        node[anchor=north west]{U};

    \tdplotsetrotatedthetaplanecoords{225};
    \tdplotdrawarc[zenazi,-stealth,tdplot_rotated_coords]
        {(0,0,0)}{0.8}{0}{40}{anchor=south west}{$\theta$};
    \draw[base,dashed,tdplot_rotated_coords] (0,0,0) -- (0,0.55,0);


    \tdplotsetrotatedcoordsorigin{(O)};
    \tdplotsetthetaplanecoords{\ra};
    \draw[base,dashed,tdplot_rotated_coords] (\clen,0,0) arc (0:90:\clen);
    \draw[base,dashed,tdplot_rotated_coords] (0,0,0) -- (0,\clen,0);
    \tdplotdrawarc[equa,-stealth,tdplot_rotated_coords]
        {(0,0,0)}{\clen}{90}{\dec}{anchor=west}{$\delta$};
    \tdplotdrawarc[equa,-stealth]
        {(O)}{\clen}{0}{\ra}{anchor=north}{$\alpha$};


\end{tikzpicture}

    \caption{Relationship between the WGS84 (orange), ENU (green),
             Zenith azimuth (pink), and clock (teal) and Equatorial
             (red) coordinate systems. $\vernal$ points to the vernal
             equinox, the point where the ecliptic and celestial equator
             cross in the Aries zodiac. WGS84 defines the position of
             the observer, LST defines the time of observation, ENU is
             the local coordinate system (only East is shown because it
             is used for the azimuth), and zenith and azimuth give the
             direction of shower origin. Combined these give the same
             position on the sky as the equatorial coordinates.}
    \label{fig:wgs84_zenazi_lst_equatorial}
\end{figure}


\section{\corsika}
\label{sec:corsika}

\corsika is an extensive air shower simulation program which we use. It
uses coordinate systems which are slightly different from the ones we
employ \cite{heck:2013aa}. The \sapphire framework provides a script to
convert \corsika output to \hdf format. During this conversion the
coordinates are transformed to the \hisparc coordinate system. The
following sections describe these transformations.

\begin{figure}
    \centering
    \tdplotsetmaincoords{70}{-10}

\pgfmathsetmacro{\rlen}{.8}
\pgfmathsetmacro{\zenith}{90-45}  % theta
\pgfmathsetmacro{\azimuth}{40}  % phi

\begin{tikzpicture}[scale=5,thick,tdplot_main_coords]

    \definecolor{base}{RGB}{128,128,128}; % grey
    \definecolor{wgs84}{RGB}{255,153,0}; % orange
    \definecolor{ecef}{RGB}{13,98,153}; % blue
    \definecolor{enu}{RGB}{0,184,0}; % green
    \definecolor{hori}{RGB}{200,20,200}; % pink
    \definecolor{corsika}{RGB}{151,115,0}; % brown

    % coordinates
    \coordinate (O) at (0,0,0);
    \tdplotsetcoord{P}{\rlen}{\zenith}{\azimuth};
    \tdplotsetcoord{P'}{\rlen}{180-\zenith}{180 + \azimuth};

    % base
    \draw[base] (O) circle (\rlen);
    \draw[base] (-\rlen*.8,-\rlen*.8,0) node[anchor=west,rotate=-10] {};
    \tdplotsetthetaplanecoords{\azimuth};
    \draw[base,dashed,tdplot_rotated_coords] (\rlen,0,0) arc (0:90:\rlen);
    \draw[base,dashed] (P) -- (Pxy);
    \draw[base,dashed] (P') -- (P'xy);
    \draw[base,dashed,tdplot_rotated_coords] (0,-\rlen,0) -- (0,\rlen,0);

    % Horizontal
    \tdplotdrawarc[hori,-stealth]
        {(O)}{0.3}{0}{\azimuth}{anchor=west}{$\phi$};
    \tdplotsetthetaplanecoords{\azimuth};
    \tdplotdrawarc[hori,-stealth,tdplot_rotated_coords]
        {(0,0,0)}{0.2}{0}{\zenith}{anchor=south}{$\theta$};

    % ENU
    \draw[enu,-stealth] (O) -- (1,0,0) node[anchor=west]{East};
    \draw[enu,-stealth] (O) -- (0,1,0) node[anchor=south east]{North};
    \draw[enu,-stealth] (O) -- (0,0,1) node[anchor=south east]{Up};

    % CORSIKA
    \tdplotdrawarc[corsika,-stealth]
        {(O)}{0.3}{90}{\azimuth+180}{anchor=south}{$\Phi$};
    \tdplotsetthetaplanecoords{\azimuth};
    \tdplotsetrotatedcoordsorigin{(P)};
    \tdplotdrawarc[corsika,-stealth,tdplot_rotated_coords]
        {(0,0,0)}{0.2}{180}{180+\zenith}{anchor=north}{$\Theta$};

    \draw[corsika,->,dashed,-stealth] (O) -- (-1,0,0) node[anchor=east]{$y$};
    \draw[corsika,->,dashed,-stealth] (O) -- (0,1,0) node[anchor=south west]{$x$};
    \draw[corsika,->,dashed,-stealth] (O) -- (0,0,1) node[anchor=south west]{$z$};

    % star
    \draw[-stealth] (O) -- (P) node[anchor=south west] {$\star$};
    \draw[base,-stealth] (O) -- (P');

\end{tikzpicture}

    \caption{Relationship between the ENU (green), Zenith Azimuth (pink),
             and \corsika (brown) coordinates.}
    \label{fig:enu_corsika}
\end{figure}


\subsection{Geographic}

\corsika defines positions on the ground (or observation level) relative
to the point where the shower axis intersects the observation level.
Positive x axis points to magnetic North, positive y axis to the West,
and the z axis upwards. This is shown by the dashed brown lines in
\figref{fig:enu_corsika}.

The formulae to convert \corsika to ENU are:

\begin{equation}
    \begin{array}{l}
        x_{\mathrm{ENU}} = -y_{\mathrm{CORSIKA}} \ , \\
        y_{\mathrm{ENU}} = x_{\mathrm{CORSIKA}} \ , \\
        z_{\mathrm{ENU}} = z_{\mathrm{CORSIKA}} \ . \\
    \end{array}
\end{equation}


\subsection{Celestial}

\corsika looks from the point of view of the shower, so not the
direction it came form, but the direction it goes to. The \corsika
zenith angle ($\Theta$) is identical to the \hisparc zenith ($\theta$),
\SI{0}{\radian} is a shower from the zenith and \SI{\pi / 2}{\radian} is
a horizontal shower. The \corsika azimuth angle ($\Phi$) differs from
the \hisparc definition of azimuth ($\phi$). The \corsika azimuth is the
angle the shower is heading to with respect to the North, while the
\hisparc azimuth is the angle the shower is coming from with respect to
the East. $\Phi = \SI{0}{\radian}$ is a shower heading towards North, so
coming from South, which we would define as $\phi = \SI{-\pi /
2}{\radian}$. The (positive) rotation of the angle is in the same
direction, from North to West. This is shown by the brown arced lines in
\figref{fig:enu_corsika}.

The formulae to convert the \corsika direction to \hisparc zenith and
azimuth are:

\begin{equation}
    \begin{array}{l}
        \theta = \Theta \ , \\
        \phi = \Phi - \frac{\pi}{2} \ . \\
    \end{array}
\end{equation}


\section{Example}
\label{sec:example}

An example will now be given for a real \hisparc event detected by
station 503.


\subsection{Measured}

From the \gps of station 503 we know its geographical location in WGS84,

\begin{equation}
    (\varphi, \lambda, h) = (\SI{52.3562600}{\degree},
                             \SI{4.9529440}{\degree},
                             \SI{51.4}{\meter}) \ .
\end{equation}

The location of its four detectors have been measured in compass
coordinates (see \figref{fig:enu_compass})

\begin{center}
    \begin{tabular}{ l c c c c }
        \toprule
        Detector & $\alpha$ [\si{\degree}] & $r$ [\si{\meter}] &
        $\beta$ [\si{\degree}] & $z$ [\si{\meter}] \\
        \midrule
        1 & 315 & 8.97 & 135 & 0 \\
        2 & 315 & 3.15 & 135 & 0 \\
        3 & 225 & 5.09 & 225 & 0 \\
        4 &  45 & 4.89 & 225 & 0 \\
        \bottomrule
    \end{tabular}
\end{center}
                        
An event was detected on \gps timestamp
\SI{1333018296870008589}{\nano\second}. The relative arrival times in
each detector are \SIlist{0;2.5;5;12.5}{\nano\second}.


\subsection{Conversions}

To reconstruct the direction of the shower we use \sapphire. First it
converts the detector compass coordinates to ENU (relative to the \gps)

\begin{center}
    \begin{tabular}{ l c c c }
        \toprule
        Detector & $x$ [\si{\meter}] & $y$ [\si{\meter}] & $z$ [\si{\meter}] \\
        \midrule
        1 & -6.34 &  6.34 & 0 \\
        2 & -2.23 &  2.23 & 0 \\
        3 & -3.60 & -3.60 & 0 \\
        4 &  3.46 &  3.46 & 0 \\
        \bottomrule
    \end{tabular}
\end{center}

The local zenith and azimuth direction is then reconstructed using the
relative arrival times and the ENU positions. In this case

\begin{equation}
    \begin{array}{l}
        \theta = \SI{0.3818}{\radian} \, \\
        \phi = \SI{3.0030}{\radian} \ ,
    \end{array}
\end{equation}

\noindent which is equal to these horizontal coordinates

\begin{equation}
    \begin{array}{l}
        a = \SI{1.1890}{\radian} \ , \\
        A = \SI{-1.4322}{\radian} \ .
    \end{array}
\end{equation}

The \gps timestamp represents the date 29 March 2012 at 10:51:21. On
that date 15 leap seconds were in effect. The corresponding UTC
timestamp, Julian Date, and GMST are

\begin{equation}
    \begin{array}{l}
        \mathit{UTC} = \SI{1333018281870008589}{\nano\second} \ , \\
        \mathit{JD} = \SI{2456015.952336}{\day} \ , \\
        \mathit{GMST} = \SI{23.3389}{\hour} = 23^\mathrm{h}20^\mathrm{m}20^\mathrm{s} \ . \\
    \end{array}
\end{equation}

Using the longitude of the station the Local Sidereal Time can be
calculated from the GMST

\begin{equation}
    \mathit{LST} = \SI{23.6691}{\hour} = 23^\mathrm{h}40^\mathrm{m}9^\mathrm{s} \ .
\end{equation}

Now the equatorial coordinates can be calculated

\begin{equation}
    \begin{array}{l}
        \alpha = \SI{5.5848}{\radian} = 21^\mathrm{h}19^\mathrm{m}57^\mathrm{s} \ , \\
        \delta = \SI{0.8730}{\radian} = \ang{50;1;} \ .
    \end{array}
\end{equation}


\section{Acknowledgements}

We want to thank Dr. J.J.M. Steijger for his careful checking of our
methods.
